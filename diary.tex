\documentclass{diary}
\usepackage{xeCJK} %使用XeLatex构建

\title{日记模板 \\{\small Plain Life and Wonderful Moments}}
\author{岚峰}

%指定起止日期
\BeginAt[year]{month}{day}
\EndAt[year]{month}{day}

%指定字体
\setCJKmainfont{华文行楷}
\setmainfont{Times New Roman}

\begin{document}
\maketitle

\Address[中国][北京]
\Date{2024}{12}{18}[\sun]%
\verb|\|Date\verb|{}{}{}[]|前三个参数分别指定年月日, 第四个参数为天气或心情,要用中括号\verb|[]|, 默认值为\verb|\sunny|, 即打印多云图标,可在.cls中\verb|\def\diary@weather{}|修改默认值。如果省略\verb|[]|或\verb|[]|内不填内容,跟随前一天填写的图标。

\verb|\|Date\verb|{}{}{}[]|后面必须要有\%,且命令与日记内容之间不能有空行,否则会引入不必要的空格和空行。

\verb|\Address[][]|两个参数设置地址,默认值为江苏南京,可在.cls中将
\verb|\newcommandx{\Address}[2][]{}|中的江苏和南京修改为其他省份城市。不带参数命令\verb|\Address|与\verb|\Address[][]|缺省两个参数显示效果相同。且可以缺省一个参数,\verb|\Address[][沭阳]|显示江苏沭阳。

\verb|\setCJKmainfont{}| 设置中文字符的字体。

\verb|\setmainfont{}| 设置文档中除中文字符以外的其他字符的字体。

\Address[][沭阳]
\Date{2025}{2}{26}[]%
\Theme[滕王阁序]
豫章故郡,洪都新府。星分翼轸,地接衡庐。襟三江而带五湖,控蛮荆而引瓯越。物华天宝,龙光射牛斗之墟;人杰地灵,徐孺下陈蕃之榻。

\Address[][]
\Date{2025}{9}{7}[\winkSmile]%
披绣闼,俯雕甍,山原旷其盈视,川泽纡其骇瞩。闾阎扑地,钟鸣鼎食之家;舸舰弥津,青雀黄龙之舳。云销雨霁,彩彻区明。落霞与孤鹜齐飞,秋水共长天一色。渔舟唱晚,响穷彭蠡之滨;雁阵惊寒,声断衡阳之浦。

\end{document}
% 天气指令
% \sun \sunny \hot \windy \clouds \snow \snowy \snowyRainy 
% \rainy \rainySunny \rainyThunder \thunderLight \hail \dust
% 心情指令
% \confused \sad \simpleSmile \slightlySmile \angry \crying \dizzy \rage 
% \winkSmile \stuckOutTongue \rollingEyes \confounded \expressionless 
% \heartEyes \laughing \nerdSmile \openMouth \smirk \worried \astonished

