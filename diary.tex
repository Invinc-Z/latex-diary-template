\documentclass{diary}
\usepackage{xeCJK} %使用XeLaTeX构建

\title{日记模板 \\{\small Plain Life and Wonderful Moments}}
\author{Invinc-Z}

%指定起止日期
\BeginAt[2025]{01}{01}   % 需手动指定年月日
\EndAt[\getYear]{\getMonth}{\getDay} % 最后一次编译当天的日期 可不修改

%指定字体
\setCJKmainfont{华文行楷}
\setmainfont{Times New Roman}

\begin{document}
\maketitle

\Address[中国][北京]
\Date{2025}{1}{1}[\sun]%
\verb|\|Date\verb|{}{}{}[]|前三个参数分别指定年月日, 第四个参数为天气或心情,要用中括号\verb|[]|, 默认值为\verb|\sunny|, 即打印多云图标,可在.cls中\verb|\def\diary@weather{}|修改默认值。如果省略\verb|[]|或\verb|[]|内不填内容,跟随前一天填写的图标。

\verb|\|Date\verb|{}{}{}[]|后面必须要有\%,且命令与日记内容之间不能有空行,否则会引入不必要的空格和空行。

\verb|\Address[][]|两个参数设置地址,默认值为江苏南京,可在.cls中将
\verb|\newcommandx{\Address}[2][]{}|中的江苏和南京修改为其他省份城市。不带参数命令\verb|\Address|与\verb|\Address[][]|缺省两个参数显示效果相同。且可以缺省一个参数,\verb|\Address[][沭阳]|显示江苏沭阳。

\verb|\setCJKmainfont{}| 设置中文字符的字体。

\verb|\setmainfont{}| 设置文档中除中文字符以外的其他字符的字体。

\verb|\Theme{}| 为每一天设定一个主题,格式为居中。

星期几根据日期值自动计算。


\Address[][沭阳]
\Date{2025}{2}{26}[]%
\Theme{滕王阁序} \par
豫章故郡,洪都新府。星分翼轸,地接衡庐。襟三江而带五湖,控蛮荆而引瓯越。物华天宝,龙光射牛斗之墟;人杰地灵,徐孺下陈蕃之榻。

披绣闼,俯雕甍,山原旷其盈视,川泽纡其骇瞩。闾阎扑地,钟鸣鼎食之家;舸舰弥津,青雀黄龙之舳。云销雨霁,彩彻区明。落霞与孤鹜齐飞,秋水共长天一色。渔舟唱晚,响穷彭蠡之滨;雁阵惊寒,声断衡阳之浦。

\Address[][]
\Date{2025}{3}{28}[\winkSmile]%
\Theme{今日小结} \par
专注时长:5 h  \par
\quad \par
今日完成:\par
1.  \par
2.  \par
3.  \par
\quad \par
明日目标:\par
1. \par
2. \par
3. \par


\end{document}
%=================================天气指令==========================================%
% \sun 晴天太阳                  \sunny 晴朗                       \hot 炎热
% \windy 大风		  			   \clouds 多云  					   \snow 雪 
% \snowy 持续降雪  				 \snowyRainy 雨夹雪                \rainy 雨天 
% \rainySunny 太阳雨             \rainyThunder 雷阵雨 			 \thunderLight 闪电(无雨)  
% \hail 冰雹		               \dust 沙尘或雾霾
%
%=================================心情指令==========================================%
% \confused 困惑			       \sad 悲伤(难过或沮丧)             \simpleSmile 简单微笑
% \slightlySmile 淡淡微笑        \angry 生气                        \crying 哭泣(大哭或极度悲伤)
% \dizzy 晕眩                    \rage 暴怒                         \winkSmile 眨眼笑
% \stuckOutTongue 吐舌(调皮搞怪) \rollingEyes 翻白眼(无语或不耐烦) \confounded 苦恼
% \expressionless 无表情         \heartEyes 爱心眼(喜爱或迷恋)      \laughing 开怀大笑
% \nerdSmile 书呆子笑(呆萌或学霸) \openMouth 张嘴(惊讶或震惊)      \smirk 得意(狡猾或自满)
% \worried 担忧(忧虑或紧张)      \astonished 震惊(极度惊讶或目瞪口呆)

